\documentclass[a4paper,12pt]{article}
\usepackage[utf8]{inputenc}
\usepackage[T1]{fontenc}
\usepackage[russian]{babel}
\usepackage{amsmath}
\usepackage{amsfonts}
\usepackage{amssymb}
\usepackage{graphicx}
\usepackage{hyperref}
\usepackage{mathptmx}
\usepackage[margin=1in]{geometry}

\title{Practicum 3}

\begin{document}

\maketitle % Добавлено для отображения заголовка

\section{Базовые характеристики графов} % Исправлено с Markdown на LaTeX

\subsection{Распределение степеней вершин} % Исправлено

Если распределение степеней вершин имеет вид степенного закона \( f(k) \sim k^{-\gamma} \) (распределение Парето), то говорят, что сеть является scale-free. Это название идет от свойства масштабной инвариантности степенного закона (инвариантность формы закона относительно \( k \to \alpha \)) --- ни один из масштабов (имеется в виду масштаб связности, в нашем случае степени вершин) не доминирует в сети, и чем больше масштаб, тем больше вероятность его обнаружить. Это приводит к характерной особенности scale-free сетей --- наличию в них хабов.

В файле \texttt{degree\_distributions.py} мы рассматриваем два графа: сеть наиболее загруженных аэропортов США и граф Эрдеша-Реньи. Первый является характерным примером scale-free сети --- действительно, хорошо известно, что некоторые аэропорты выполняют функцию хабов. Далеко не все графы обладают таким свойством. На самом деле интерес к scale-free сетям появился только к концу 90-х, когда стало очевидно, что WWW сеть содержит в себе много хабов и, следовательно, ее свойства принципиально отличаются от других случайных сетей, таких, как, например, граф Эрдеша-Реньи.

\section{Меры центральности}

Центральность вершины характеризует ее важность в данном графе. Мера центральности, следовательно, является вещественнозначным выражением центральности. Так как «важность» может пониматься по-разному в различных задачах, существует несколько конкурирующих мер:
\begin{itemize}
    \item степень вершины,
    \item степень близости (closeness centrality),
    \item степень посредничества (betweenness centrality),
    \item степень влиятельности (eigenvector centrality),
    \item PageRank.
\end{itemize}

В файле \texttt{centrality\_measures.py} мы реализуем и анализируем степени близости, посредничества и влиятельности.

\subsection{Степень близости (closeness centrality)} % Исправлена опечатка

Для вершины \( v \in V \) графа \( G = (V, E) \):
\[
C(v) = \frac{1}{\sum_{u \in V} \text{dist}(v, u)},
\]
где \( \text{dist}(\cdot, \cdot) \) является функцией кратчайшего расстояния. При сравнении степеней близости в различных графах эту меру нормализуют коэффициентом \( 1/|V| \).

\subsection{Степень посредничества (betweenness centrality)}

Эта мера характеризует то, насколько данная вершина часто участвует во взаимодействии двух других вершин. Более конкретно, насколько часто она является мостом в кратчайшем пути между этими вершинами:
\[
C(v) = \sum_{s \neq t \neq v \in V} \frac{\sigma(s, t | v)}{\sigma(s, t)},
\]
где \( \sigma(s, t | v) \) есть количество кратчайших путей из \( s \) в \( t \), проходящих через \( v \), а \( \sigma(s, t) \) есть общее количество кратчайших путей из \( s \) в \( t \).

\subsection{Степень влиятельности (eigenvector centrality)}

Меры, характеризующие влиятельность, определяют ее рекурсивно через влиятельность соседей. Подобные рекурсивные формулировки часто приводят к задаче о собственных значениях.

Например, мы можем задать степень влиятельности данного узла как сумму степеней влиятельности соседей с некоторым коэффициентом \( \lambda \):
\[
C(v) = \frac{1}{\lambda} \sum_{u \in \text{Nei}(v)} C(u) = \frac{1}{\lambda} \sum_{u \in V} a_{v, u} C(u),
\]
где \( a_{v, u} \) --- элементы матрицы смежности \( \boldsymbol{A} \).

Сложив все значения \( C(v) \) в вектор \( \boldsymbol{c} \), мы получаем задачу о собственных значениях матрицы смежности:
\[
\boldsymbol{A} \boldsymbol{x} = \lambda \boldsymbol{x}.
\]
Для данного собственного числа собственный вектор будет соответствовать степеням влиятельности каждой вершины. Так как степени влиятельности должны быть неотрицательными, то по теореме Фробениуса-Перрона единственное подходящее собственное число является наибольшим по модулю собственным числом (оно же по этой теореме будет вещественным и строго положительным).

\end{document}